\documentclass[11pt]{article}
\usepackage{fullpage}
\usepackage[left=1in,top=1in,right=1in,bottom=1in,headheight=3ex,headsep=3ex]{geometry}

\newcommand{\blankline}{\quad\pagebreak[2]}

\title{ECON 2250: Statistics for Economists}
\author{Jim Crozier}
\date{Fall 2022}

\usepackage[sc]{mathpazo}
\linespread{1.05}         % Palatino needs more leading (space between lines)
\usepackage[T1]{fontenc}

\usepackage[mmddyyyy]{datetime}% http://ctan.org/pkg/datetime
\usepackage{advdate}% http://ctan.org/pkg/advdate
\newdateformat{syldate}{\twodigit{\THEMONTH}/\twodigit{\THEDAY}}
\newsavebox{\MONDAY}\savebox{\MONDAY}{Mon}% Mon

\newcommand{\week}[1]{%
%  \cleardate{mydate}% Clear date
% \newdate{mydate}{\the\day}{\the\month}{\the\year}% Store date
  \paragraph*{\kern-2ex\quad #1, \syldate{\today} - \AdvanceDate[4]\syldate{\today}:}% Set heading  \quad #1
%  \setbox1=\hbox{\shortdayofweekname{\getdateday{mydate}}{\getdatemonth{mydate}}{\getdateyear{mydate}}}%
  \ifdim\wd1=\wd\MONDAY
    \AdvanceDate[7]
  \else
    \AdvanceDate[7]
  \fi%
}



\usepackage{setspace}
\usepackage{multicol}
%\usepackage{indentfirst}
\usepackage{fancyhdr,lastpage}
\usepackage{url}
\pagestyle{fancy}
\usepackage{hyperref}
\usepackage{lastpage}
\usepackage{amsmath}
\usepackage{layout}   
\lhead{}
\chead{}
\rhead{\footnotesize Statistics for Economics  -- Fall 2022}
\lfoot{}
\cfoot{\small \thepage/\pageref*{LastPage}}
\rfoot{}

\usepackage{array, xcolor}
\usepackage{color,hyperref}
\definecolor{clemsonorange}{HTML}{B3A369}
\hypersetup{colorlinks,breaklinks,
            linkcolor=clemsonorange,urlcolor=clemsonorange,
            anchorcolor=clemsonorange,citecolor=black}




\begin{document}


\maketitle

\blankline

\begin{tabular*}{.93\textwidth}{@{\extracolsep{\fill}}lr}


  E-mail: \texttt{rcrozier3@gatech.edu} & Repo: \href{http://github.com/jimcrozier/gt\_econ2250}{\tt\bf github.com/jimcrozier/gt\_econ2250}  \\

Office Hours: TBD  &  Class Hours: TBD \\


Office: TBD \\
TA: TBD \\
&  \\
\hline
\end{tabular*}

\vspace{10 mm}

\section*{Course Description}

In this class we will cover an introduction into statistical computing which is a cornerstone to economic research. The course will be about one half statistical concepts [e.g. counts, summary statistics, and the basics of building statistical models], and one half practical coding skills that are necessary for estimating these concepts. I assume that the incoming student has a general knowledge of mathematical principals, most importantly familiarity with manipulating algebraic equations as well as basic probability theory and intuition. I also assume that the student has a basic understanding of python and notebooks. Although each of these subjects will be taught from the ground up, if this is the first encounter with any of these concepts then I expect a pretty steep learning curve. 


\section*{Course Objectives}
\begin{enumerate}
\item Introductory statistics proficiency (counts, summary stats, linear models, significance tests).
\item Data munging and basic SQL syntax. 
\item Introductory statistical coding python proficiency.
\item Short video presentation of notebooks skills.
\item Strengthen voice in expressing statistical analysis. 
\end{enumerate}

\section*{Texts}

\begin{itemize}
\item Cunningham, Scott. 2021. Causal Inference: The Mixtapes.  \href{https://mixtape.scunning.com/}{\tt\bf  Online version is fine.} 

\item Hastie, Tibshirani, Friedman. 2009.  The Elements of Statistical Learning  \href{https://hastie.su.domains/ElemStatLearn/}{\tt\bf  Online version is fine.} 

\end{itemize} 


\section*{Honor Code and Plagiarism}
You are expected to follow the Georgia Institute of Technology Honor Code at all times. 
For any questions involving these or any other Academic Honor Code issues, please consult me or
http://www.honor.gatech.edu.


\subsection*{Grading Policy}
\begin{itemize}
\item \underline{\textbf{40\%}} of your grade will be determined by your homework videos. There will be a total of 10 homework videos graded (see below) and I will drop the lowest two grades and average the others. 
\item \underline{\textbf{25\%}} of your grade will be determined by the midterm grade. 
\item \underline{\textbf{35\%}} of your grade will be determined by a final exam.
\item \underline{\textbf{5\%}} extra credit (see below). 
\end{itemize}

\subsubsection*{Homework Videos}

Each week that we do not an exam (see schedule) you will have to submit a *no longer than* 2 minute video of your notebook assignment for that week, along with the notebook. You are welcome to work with others on the homework assignment, but your video and your code/notebook will need to be your own. The videos will reviewed by the TA and given an 0-100\% grade. This is homework so my intention here is that a good shot at the work, and admitting what you can't get should be a 100\%. The general idea here is to help you build your voice of your analysis. The grading rubric for the assignment are:
  \begin{itemize}
\item \underline{\textbf{50\%}} does the notebook achieve the directions, or at least make a solid stab at and explain what they couldn't get in their video.  
	\item \underline{\textbf{40\%}} does the video express the ideas of the analysis in a straightforward that explains what the author believes and how confident they are in the results. Just screen cast the notebook and explain what you did, and even what you didn't understand. 
\item \underline{\textbf{10\%}} is the video under 2 minutes. Not that faster is always better, but there are a lot of videos to grade and the point of this whole thing is learn to trust your voice, explain how much you believe it, and also state stuff that you didn't get. If you can do that in 30 seconds, all the better for all of us.   
\end{itemize}

\subsubsection*{Exams}
Midterm and final will be cumulative and cover statistical concepts and test the ability to read output of statistical code and explain the outputs in well formulated arguments that address central tendency and dispersion.


\subsubsection*{Extra Credit}
The repo for this class will include links to *open source* videos, lectures, papers, datasets, code that link to each lecture. If you *CREATE A \href{https://docs.github.com/en/pull-requests/collaborating-with-pull-requests/proposing-changes-to-your-work-with-pull-requests/creating-a-pull-request}{\tt\bf  PULL REQUEST} * to add an interesting contribution you can earn 1 point to your grade, up to 5 points.


*Open source is a minimum of MIT license. I will NOT merge a PR if the content does not have a OS license.



\subsection*{Special Accommodations}

If you need any special accommodations due to a physical or learning disability, please let me
know during the first week of class. In order to receive the requested accommodations you will
need to obtain a form from the Access Disabled Assistance Program for Tech Students (ADAPTS)
and give me this form. The ADAPTS Office is located in the Smithgall Student Services Building,
Suite 210 and the website is http://www.adapts.gatech.edu.
Also, if you will be missing any classes for religious holidays or other events, let me know as soon
as you know you will be missing class. You will still be required to know the material from that
class period.

\subsection*{Policy on Children in Class (adopted from \@guygrossman)}

\begin{enumerate}
\item You and your baby are welcome in class anytime.
\item For older children and babies, I understand that minor illnesses and unforeseen disruptions in childcare often put parents in the position of having to choose between missing class to stay home with a child and leaving them with someone you or the child does not feel comfortable with. While this is not meant to be a long-term childcare solution, occasionally bringing a child to class in order to cover gaps in care is perfectly acceptable.
\item I ask that all students work with me to create a welcoming environment that is respectful of all forms of diversity, including diversity in parenting status.
\item In all cases where babies and children come to class, I ask that you sit close to the door so that if your little one needs special attention and is disrupting learning for other students, you may step outside until their need has been met. Non-parents in the class, please reserve seats near the door for your parenting classmates.
\item Finally, I understand that often the largest barrier to completing your coursework once you become a parent is the tiredness many parents feel in the evening once children have finally gone to sleep. The struggles of balancing school and childcare are exhausting! I hope that you will feel comfortable disclosing your student-parent status to me. This is the first step in me being able to accommodate any special needs that arise. While I maintain the same high expectations for all students in my classes regardless of parenting status, I am happy to problem solve with you in a way that makes you feel supported as you strive for school-parenting balance. 
\end{enumerate}

\subsection*{Schedule (subject to change)}
\SetDate[22/08/2022]
\week{Week 01} Load data, summary stats, simple graphs, making videos.
\week{Week 02} Continuous vs discrete, counts, descriptive stats.
\week{Week 03} Dependent vs independent, light causality. 
\week{Week 04} Probability review. 
\week{Week 05} Probability review 2: Conditional Probability.
\week{Week 06} Summation notation. 
\week{Week 07} Correlation and model introduction. 
\week{Week 08} Review and Midterm: conditional prob to correlation.
\week{Week 09} Average Treatment Effect.
\week{Week 10} Linear regression. 
\week{Week 11} Linear regression 2. 
\week{Week 12} Standard error. 
\week{Week 13} Test for significance. 
\week{Week 14} Continuous vs discrete modeling. 
\week{Week 15} Bring it all together. 
\week{Week 16} Final: ATE and linear reg. 


\end{document}
