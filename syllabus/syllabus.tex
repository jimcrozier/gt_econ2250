\documentclass[11pt]{article}
\usepackage{fullpage}
\usepackage[left=1in,top=1in,right=1in,bottom=1in,headheight=3ex,headsep=3ex]{geometry}
\usepackage[T1]{fontenc}
\newcommand{\blankline}{\quad\pagebreak[2]}
\usepackage{multirow}
\title{ECON 2250: Statistics for Economists}
\author{Jim Crozier}
\date{Fall 2022}

\usepackage[sc]{mathpazo}
\linespread{1.05}         % Palatino needs more leading (space between lines)
\usepackage[T1]{fontenc}
\newcommand{\emdash}{\nobreak-\nobreak\hskip0pt}
\usepackage[mmddyyyy]{datetime}% http://ctan.org/pkg/datetime
\usepackage{advdate}% http://ctan.org/pkg/advdate
\newdateformat{syldate}{\twodigit{\THEMONTH}/\twodigit{\THEDAY}}
\newsavebox{\MONDAY}\savebox{\MONDAY}{Mon}% Mon

\newcommand{\week}[1]{%
%  \cleardate{mydate}% Clear date
% \newdate{mydate}{\the\day}{\the\month}{\the\year}% Store date
  \paragraph*{\kern-2ex\quad #1, \syldate{\today} - \AdvanceDate[4]\syldate{\today}:}% Set heading  \quad #1
%  \setbox1=\hbox{\shortdayofweekname{\getdateday{mydate}}{\getdatemonth{mydate}}{\getdateyear{mydate}}}%
  \ifdim\wd1=\wd\MONDAY
    \AdvanceDate[7]
  \else
    \AdvanceDate[7]
  \fi%
}



\usepackage{setspace}
\usepackage{multicol}
%\usepackage{indentfirst}
\usepackage{fancyhdr,lastpage}
\usepackage{url}
\pagestyle{fancy}
\usepackage{hyperref}
\usepackage{lastpage}
\usepackage{amsmath}
\usepackage{layout}   
\lhead{}
\chead{}
\rhead{\footnotesize Statistics for Economics  -- Fall 2022}
\lfoot{}
\cfoot{\small \thepage/\pageref*{LastPage}}
\rfoot{}

\usepackage{array, xcolor}
\usepackage{color,hyperref}
\definecolor{clemsonorange}{HTML}{B3A369}
\hypersetup{colorlinks,breaklinks,
            linkcolor=clemsonorange,urlcolor=clemsonorange,
            anchorcolor=clemsonorange,citecolor=black}




\begin{document}


\maketitle

\blankline

\begin{tabular*}{.93\textwidth}{@{\extracolsep{\fill}}lr}



  E-mail: \texttt{rcrozier3@gatech.edu} \\
  Repo: \href{http://github.com/jimcrozier/gt\_econ2250}{\tt\bf github.com/jimcrozier/gt\_econ2250}  \\


  Class Hours: MWF 12:30 - 1:20 \\
  Class location: 	Van Leer E361 \\

  Office Hours: Fridays 2-3PM  \\
Office: 310 of the Rich Computer Center  \\
TA: Christopher Kitchens, email: christopher.kitchens@gatech.edu \\
&  \\
\hline
\end{tabular*}

\vspace{10 mm}

\section*{Course Description}

In this class we will cover an introduction into statistical computing which is a cornerstone to economic research. The course will be about one half statistical concepts [e.g. counts, summary statistics, and the basics of building statistical models], and one half practical coding skills that are necessary for estimating these concepts. I assume that the incoming student has a general knowledge of mathematical principals, most importantly familiarity with manipulating algebraic equations as well as basic probability theory and intuition. I also assume that the student has a basic understanding of python and notebooks. Although each of these subjects will be taught from the ground up, if this is the first encounter with any of these concepts then I expect a pretty steep learning curve. 




\section*{Course Objectives}
\begin{enumerate}
\item Introductory statistics proficiency (counts, summary stats, linear models, significance tests).
\item Introductory statistical coding python proficiency.
\item Short video presentation of notebooks skills.
\item Strengthen voice in expressing statistical analysis. 
\end{enumerate}

\section*{Texts}

You are not required to purchase these texts, the online versions are fine (although they are both great reference books)
\begin{itemize}
\item Cunningham, Scott. 2021. Causal Inference: The Mixtapes.  \href{https://mixtape.scunning.com/}{\tt\bf  Online version is fine.} 

\item Hastie, Tibshirani, Friedman. 2009.  The Elements of Statistical Learning  \href{https://hastie.su.domains/ElemStatLearn/}{\tt\bf  Online version is fine.} 


We will also use many online reference materials which will be listed in the github repo. 
\end{itemize} 


\section*{Honor Code and Plagiarism}
You are expected to follow the Georgia Institute of Technology Honor Code at all times. 
For any questions involving these or any other Academic Honor Code issues, please consult me or
http://www.honor.gatech.edu.

Notice that for homeworks, I am completely fine with students working together and using any online materials. That said in many of the homeworks you will each receive different data, and you are responsible for explaining the outcomes yourself.

All videos must be done by yourself, cover your own data (or data seed), and explain your own outputs. If your video presentation does not present your own results, or I do not feel that you can explain your code or results in a way that makes it clear that you understand how it is working you will not receive credit for that assignment. 


\section*{canvas}
Communication   about   the   course   and   other   course   materials   is   available   on   Georgia   Tech’s
Canvas system.  I will upload additional course materials and post course announcements on the
course page.  You may access Canvas at https://canvas.gatech.edu/.


\subsection*{Grading Policy}
\begin{itemize}
\item \underline{\textbf{40\%}} of your grade will be determined by your homework videos. There will be a total of 10 homework videos graded (see below) and I will drop the lowest two grades and average the others. 
\item \underline{\textbf{25\%}} of your grade will be determined by the midterm grade. 
\item \underline{\textbf{35\%}} of your grade will be determined by a final exam.
\item \underline{\textbf{5\%}} extra credit (see below). 
\end{itemize}


Your final grades transfer to the Georgia Tech
grading system as follows:

\begin{tabular}{ |p{3cm}||p{3cm}|p{3cm}|p{3cm}|  }
  \hline
  \multicolumn{4}{|c|}{Grading Rubic} \\
  \hline
 A & 4.0 &Excellent&90\%-100\%\\
  \hline
  B   & 3.0    &Good&   80\%-89.99\%\\
  \hline
C&  2.0 & Satisfactory   &70\%-79.99\%\\
\hline
D&1.0 & Passing&  60\%-69.99\%\\
\hline
  F    &0.0 & Failure& < 60\%\\
  \hline
 \end{tabular}

 I  reserve   the   right   to   increase   all   final   grades   equally   if   the   material   is   more   difficult   than
 expected, but this is unlikely.   At the end of the semester your final grade cannot be changed
 with extra credit or makeup opportunities; this is for fairness to all students.
 \subsubsection*{Pass $/$fail policy}
 If you are taking the class pass/fail, you must achieve a C grade to earn a pass in the class.
 Please reach out to me at the beginning of the semester if you are taking the class pass/fail.


\subsubsection*{Homework Videos}
Homeworks are due on Thursday morning at 10AM EST. Submit the video (youtube link) and colab notebook link to Canvas. 

\textbf{*** Because I am dropping your two lowest, I am not accepting late homework. Homework submitted at 10:01AM will not be accepted. ***}


Each week that we do not have an exam (see schedule) you will have to submit a *no longer than* 2 minute video of your notebook assignment for that week, along with the notebook. You are welcome to work with others on the homework assignment, but your video and your code/notebook will need to be your own. The videos will reviewed by the TA and given an 0-100\% grade, although final grades for the assignments will be determined by me and your talk to me if you have any issues. As always, respect and kindness are expected in all communications. This is homework so my intention here is that a good shot at the work, and admitting what you can't get should be a 100\%. The general idea here is to help you build your voice of your analysis. The grading rubric for the assignment are:
  \begin{itemize}
\item \underline{\textbf{50\%}} does the notebook achieve the directions, or at least make a solid stab at and explain what they couldn't get in their video.  
	\item \underline{\textbf{40\%}} does the video express the ideas of the analysis in a straightforward that explains what the author believes and how confident they are in the results. Just screen cast the notebook and explain what you did, and even what you didn't understand. 
\item \underline{\textbf{10\%}} is the video under 2 minutes. Not that faster is always better, but there are a lot of videos to grade and the point of this whole thing is learn to trust your voice, explain how much you believe it, and also state stuff that you didn't get. If you can do that in 30 seconds, all the better for all of us.   
\end{itemize}

Our repo will have suggestions on how to record, but you will not be reviewed on production quality, just the results. 

The point of these exercises is to get better at stats and coding, please do not attempt to just present someone's else work. It will show through in your presentations, and worst case you will not receive credit for the homework. 



\subsubsection*{Homework Videos}

\subsubsection*{Exams}
Midterm and final will be cumulative and cover statistical concepts and test the ability to read output of statistical code and explain the outputs in well formulated arguments that address central tendency and dispersion.


\subsubsection*{Extra Credit}
The repo for this class will include links to *open source* videos, lectures, papers, datasets, code that link to each lecture. If you *CREATE A \href{https://docs.github.com/en/pull-requests/collaborating-with-pull-requests/proposing-changes-to-your-work-with-pull-requests/creating-a-pull-request}{\tt\bf  PULL REQUEST} * to add an interesting contribution you can earn 1 point to your grade, up to 5 points.


*Open source is a minimum of MIT license. I will NOT merge a PR if the content does not have a OS license.



\subsection*{Freedom of expression}
As   a   faculty   member   at   Georgia   Tech,   I   respect   your   rights   to   the   freedom   of   speech   and
expression.  I am also committed to maintaining an orderly learning environment for all students
and ensuring that all facilities are used in a way that facilitate teaching, learning, and research.
Therefore, you should treat your peers and instructor respectfully in discussion.  Disagreements
are likely to happen. When they do, you are expected to disagree respectfully and to keep your
discussion focused on evidence.
Discussions in this class are expected to take place solely within the course.   Thus, statements
made during class should not be quoted on social media unless the individual being quoted has
provided their express permission.  This applies to the instructor, students, and classroom guests.
This policy is meant to protect student privacy and create a safe environment to learn.
\subsection*{Academic integrity}
“I commit to uphold the ideals of honor and integrity by refusing to betray the trust bestowed
upon me as a member of the Georgia Tech community.”
Georgia   Tech   aims   to   cultivate   a   community   based   on   trust,   academic   integrity,   and   honor.
Students   are   expected   to   act   according   to   the   highest   ethical   standards.     For   information   on
Georgia   Tech's   Academic   Honor   Code,   please   visit
\href{http://www.catalog.gatech.edu/policies/honor-code/}{\tt\bf http://www.catalog.gatech.edu/policies/honor-code/}  or 
\href{http://www.catalog.gatech.edu/rules/18/.}{\tt\bf http://www.catalog.gatech.edu/rules/18/.} 
Academic integrity is extremely important to me.
Any student suspected of cheating or plagiarizing on an exam or assignment will be reported to
the Office of Student Integrity, who will investigate the incident and identify the appropriate
penalty for violations.
\subsection*{Student-faculty expectations agreement}
It   is   important   to   strive   for   an   atmosphere   of   mutual   respect,   acknowledgement,   and
responsibility   between   faculty   members   and   the   student   body.   See
\href{http://www.catalog.gatech.edu/rules/22/.}{\tt\bf http://www.catalog.gatech.edu/rules/22/.} 
 for an articulation of some basic expectations that you
can have of me and that I have of you. In the end, simple respect for knowledge, hard work, and
cordial   interactions   will   help   build   the   environment   we   seek.   Therefore,   I   encourage   you   to
remain committed to the ideals of Georgia Tech while in this class.
\subsection*{Athletics absences}
If you are a student athlete, you must obtain statements of approved absence from the Office of
the Registra (\href{https://registrar.gatech.edu/}{\tt\bf https://registrar.gatech.edu/}) one week in advance of your planned absence.  I will
accommodate your schedule if I have enough time, so please get in touch early.
\subsection*{Accommodations for students with disabilities}
If you are a student with learning needs that require special accommodation, contact the Office
of   Disability   Services   at   (404)   894-2563   or \href{http://disabilityservices.gatech.edu/}{\tt\bf http://disabilityservices.gatech.edu/}    as   soon   as
possible to make an appointment to discuss your special needs and to obtain an accommodations
letter.  Please also e-mail me as soon as possible to set up a time to discuss your learning needs.
Your accommodations cannot be put into place until you discuss them with me. 
\subsection*{This syllabus is subject to change}
At any time for any reason, I may need to update this syllabus.  I will inform you immediately if
I make any changes.


\subsection*{Schedule (subject to change)}

\begin{itemize}
  \item \underline{\textbf{Week 1 (2022/08/22 - 2022/08/26):}} Load data, summary stats, simple graphs, making videos.
  \begin{itemize}
    \item 2022/08/22: Introductions, syllabus, colab intro.
    \item 2022/08/24: Colab deep dive, python intro.
    \item 2022/08/26: Data types, summary statistics.
  \end{itemize}
  
  \item \underline{\textbf{Week 2 (2022/08/29 - 2022/09/02):}} Continuous vs discrete, counts, descriptive stats.
  \begin{itemize}
    \item 2022/08/29: Colab and summary statistics type reveiw. 
    \item 2022/08/31: Colab deep dive, python intro, discreet vs continuous.
    \item 2022/09/01: HW1 due: Loading data into colab and making first video.
    \item 2022/09/02: Summary stats tables: mean, median, and standard deviation. 
  \end{itemize}

  \item \underline{\textbf{Week 3 (2022/09/05 - 2022/09/09):}} Dependent vs independent, light causality. 
  \begin{itemize}
    \item 2022/09/05: LABOR DAY BREAK
    \item 2022/09/07: Reveiw mean, median, standard deviation. 
    \item 2022/09/08: HW2 due: mean, median, standard deviation.
    \item 2022/09/09: Dependent vs indendepent variable, deep dive a data frame. 
  \end{itemize}


  \item \underline{\textbf{Week 4 (2022/09/12 - 2022/09/16):}} Probability review. 
  \begin{itemize}
    \item 2022/09/12: Reveiw depdentent vs independent, discuss probability. 
    \item 2022/09/14: Rules of probability. 
    \item 2022/09/15: HW3 due: depvar and covariates, light probabilty exercises.
    \item 2022/09/16: Review  of probability rules. 
  \end{itemize}

  \item \underline{\textbf{Week 5 (2022/09/19 - 2022/09/23):}} Probability review 2: Conditional Probability.
  \begin{itemize}
    \item 2022/09/19: Conditional probability, bayes rule.
    \item 2022/09/21: More of that, it's tricky. 
    \item 2022/09/22: HW4 due: some fun stuff with conditional probabilty and functions.
    \item 2022/09/23: Pull together probabilty review. 
  \end{itemize}


  \item \underline{\textbf{Week 6 (2022/09/26 - 2022/09/30):}} Summation notation. 
  \begin{itemize}
    \item 2022/09/26: We are going to have to spend some time on notation, here is where we dive  in to the summation operator.
    \item 2022/09/28: Some work on the error term to the sum of squares. 
    \item 2022/09/08: HW5 due: turn that math into code using the summation operator and functions.
    \item 2022/09/30: review of stats proofs and summation notation. We are going to need it quite a bit. 
  \end{itemize}


  \item \underline{\textbf{Week 7 (2022/10/03 - 2022/10/07):}} Correlation and model introduction. 
  \begin{itemize}
    \item 2022/10/03: (progress report) Introduction to model building and correlation of independent and dependent variables.
    \item 2022/10/07: No semester is complete with correlation vs causation, this is that lecture. 
    \item 2022/09/08: HW6 due: correlation vs causation the notebook. 
    \item 2022/10/09: Digging further into correlation. 
  \end{itemize}

  \item \underline{\textbf{Week 8 (2022/10/10 - 2022/10/14):}} Review and Midterm: conditional prob to correlation.
  \begin{itemize}
    \item 2022/10/10: Reveiw probabilty rules. 
    \item 2022/10/12: Review correlation. 
    \item 2022/10/14: MIDTERM EXAM - you got this. 
  \end{itemize}

  \item \underline{\textbf{Week 9 (2022/10/17 - 2022/10/21):}} Average Treatment Effect.
  \begin{itemize}
    \item 2022/10/17: FALL BREAK 
    \item 2022/10/19: Review of midterm material. Discusss average treatment effect. 
    \item 2022/10/21: Average treatment effect further discussion. 
  \end{itemize}

  \item \underline{\textbf{Week 10 (2022/10/24 - 2022/10/28):}} Linear regression. 
  \begin{itemize}
    \item 2022/10/24: Error term, sum of squared residual, and best fit line. 
    \item 2022/10/26: Linear regression summation notation. 
    \item 2022/10/27: HW7 due: linear regression model in sklearn.
    \item 2022/10/28: Futher discussion on linear regression model assumptions. 
    \item 2022/10/29: WITHDRAWAL DEADLINE
  \end{itemize}


  \item \underline{\textbf{Week 11 (2022/10/31 - 2022/11/04):}} Linear regression 2. 
  \begin{itemize}
    \item 2022/10/31: Reveiw of simple regression and model assumptions.
    \item 2022/11/02: Linear model predictions.
    \item 2022/11/03: HW8 due: Linear model predictions. 
    \item 2022/11/04: More talk of linear model assumptions. 
  \end{itemize}


  \item \underline{\textbf{Week 12 (2022/11/07 - 2022/11/11):}} Standard error. 
  \begin{itemize}
    \item 2022/11/07: Dispersion, and why it matters. We're really starting to pull everything together now.
    \item 2022/11/09: Calulating standard errors. 
    \item 2022/11/10: HW9 due: linear model dispersion notebook. 
    \item 2022/11/11: More time on standard errors.
  \end{itemize}


  \item \underline{\textbf{Week 13 (2022/11/14 - 2022/11/18):}} Test for significance. 
  \begin{itemize}
    \item 2022/11/14: What is significance?
    \item 2022/11/16: Calcuating significance and expressing in your results. 
    \item 2022/11/17: HW10 due: tie it all together with model and significance testing. 
    \item 2022/11/18: Further digressions on testing significance. 
  \end{itemize}


  \item \underline{\textbf{Week 14 (2022/11/21 - 2022/11/25):}} Continuous vs discrete modeling. 
  \begin{itemize}
    \item 2022/11/21: Bonus material: continuous vs discreete model, and why you would care. 
    \item 2022/11/23: THANKSGIVING BREAK
    \item 2022/11/25: THANKSGIVING BREAK
  \end{itemize}


  \item \underline{\textbf{Week 15 (2022/11/28 - 2022/12/02):}} Bring it all together. 
  \begin{itemize}
    \item 2022/11/28: Review week (nothing new): Working together we'll take a continuous and discreet problem: tell summary stats
    \item 2022/11/30: Continue above with build a linear model, work out the math, code it up
    \item 2022/12/02: Continue the above show model fit and significance tests. 
  \end{itemize}

  \item \underline{\textbf{Week 16 (2022/12/05 - 2022/12/09):}} Final Exam
  \begin{itemize}
    \item 2022/12/28: Review
    \item 2022/12/30: READING DAY
    \item 2022/12/02: Final exam
  \end{itemize}
\end{itemize}




\end{document}
