\documentclass[11pt]{article}
\usepackage{fullpage}
\usepackage[left=1in,top=1in,right=1in,bottom=1in,headheight=3ex,headsep=3ex]{geometry}

\newcommand{\blankline}{\quad\pagebreak[2]}

\title{ECON 2250: Statistics for Economists}
\author{Jim Crozier}
\date{Fall 2022}

\usepackage[sc]{mathpazo}
\linespread{1.05}         % Palatino needs more leading (space between lines)
\usepackage[T1]{fontenc}

\usepackage[mmddyyyy]{datetime}% http://ctan.org/pkg/datetime
\usepackage{advdate}% http://ctan.org/pkg/advdate
\newdateformat{syldate}{\twodigit{\THEMONTH}/\twodigit{\THEDAY}}
\newsavebox{\MONDAY}\savebox{\MONDAY}{Mon}% Mon

\newcommand{\week}[1]{%
%  \cleardate{mydate}% Clear date
% \newdate{mydate}{\the\day}{\the\month}{\the\year}% Store date
  \paragraph*{\kern-2ex\quad #1, \syldate{\today} - \AdvanceDate[4]\syldate{\today}:}% Set heading  \quad #1
%  \setbox1=\hbox{\shortdayofweekname{\getdateday{mydate}}{\getdatemonth{mydate}}{\getdateyear{mydate}}}%
  \ifdim\wd1=\wd\MONDAY
    \AdvanceDate[7]
  \else
    \AdvanceDate[7]
  \fi%
}



\usepackage{setspace}
\usepackage{multicol}
%\usepackage{indentfirst}
\usepackage{fancyhdr,lastpage}
\usepackage{url}
\pagestyle{fancy}
\usepackage{hyperref}
\usepackage{lastpage}
\usepackage{amsmath}
\usepackage{layout}   
\lhead{}
\chead{}
\rhead{\footnotesize Statistics for Economics  -- Fall 2022}
\lfoot{}
\cfoot{\small \thepage/\pageref*{LastPage}}
\rfoot{}

\usepackage{array, xcolor}
\usepackage{color,hyperref}
\definecolor{clemsonorange}{HTML}{B3A369}
\hypersetup{colorlinks,breaklinks,
            linkcolor=clemsonorange,urlcolor=clemsonorange,
            anchorcolor=clemsonorange,citecolor=black}




\begin{document}


\maketitle

\blankline

\begin{tabular*}{.93\textwidth}{@{\extracolsep{\fill}}lr}


  E-mail: \texttt{rcrozier3@gatech.edu} & Repo: \href{http://github.com/jimcrozier/gt\_econ2250}{\tt\bf github.com/jimcrozier/gt\_econ2250}  \\

&  \\
\hline
\end{tabular*}

\vspace{10 mm}

\section*{Course Description}

In this class we will cover an introduction into statistical computing which is a cornerstone to economic research. The course will be about one half statistical concepts [e.g. counts, summary statistics, and the basics of building statistical models], and one half practical coding skills that are necessary for estimating these concepts. I assume that the incoming student has a general knowledge of mathematical principals, most importantly familiarity with manipulating algebraic equations as well as basic probability theory and intuition. I also assume that the student has a basic understanding of python and notebooks. Although each of these subjects will be taught from the ground up, if this is the first encounter with any of these concepts then I expect a pretty steep learning curve. 



\end{document}
